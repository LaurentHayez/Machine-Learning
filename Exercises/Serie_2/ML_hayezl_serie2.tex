\documentclass[fontsize=12pt, usenames, dvipsnames, headinclude, headsepline, footinclude, footsepline]{scrartcl}

\usepackage[utf8]{inputenc}
\usepackage[T1]{fontenc}
\usepackage{graphicx, wrapfig}
\usepackage{lmodern}
\usepackage{color, colortbl}
\usepackage{xcolor}
\usepackage{amsmath, amssymb, mathrsfs, amsthm, thmtools, MnSymbol}
\usepackage[framemethod=tikz]{mdframed}
\usepackage{pgf, pgfplots, tikz, pst-solides3d}
\usetikzlibrary{cd} %To draw commutative diagrams
\usetikzlibrary{calc}
\usetikzlibrary{arrows}
\usetikzlibrary{shapes}
\usetikzlibrary{lindenmayersystems, arrows.meta}
\usetikzlibrary{decorations.pathreplacing}
\usetikzlibrary{automata}
\usepackage{scrlayer-scrpage}  % header and footer for KOMA-Script
\usepackage{hyperref}
\usepackage{todonotes}
\usepackage{listings}
\usepackage[inline]{enumitem}
\usepackage{booktabs}
\usepackage{multirow}
\usepackage[english]{babel}


\newcommand{\N}{\mathbb{N}}
\newcommand{\Q}{\mathbb{Q}} 
\newcommand{\R}{\mathbb{R}}
\newcommand{\Z}{\mathbb{Z}}
\newcommand{\F}{\mathbb{F}}
\newcommand{\T}{\mathcal{T}}
\renewcommand{\P}{\mathbb{P}}
\renewcommand{\S}{\mathbb{S}}
\newcommand{\bw}{\bigwedge}
\newcommand{\Fa}{\F(A)} 
\newcommand{\C}{\mathbb{C}}
\newcommand{\K}{\mathbb{K}}
\renewcommand{\epsilon}{\varepsilon}
\renewcommand{\phi}{\varphi}
\renewcommand{\emph}{\textbf}
\newcommand{\im}{\mathrm{im}}


\synctex=1


%%%%%%%%	Définitions des environnements de théorèmes	%%%%%%%%
%----- ENVIRONNEMENT POUR LES EXERCICES ----%
\declaretheoremstyle[
  spaceabove=0pt, spacebelow=0pt, headfont=\normalfont\bfseries\scshape,
    notefont=\mdseries, notebraces={(}{)}, headpunct={. }, headindent={},
    postheadspace={ }, postheadspace=4pt, bodyfont=\normalfont\itshape
]{defstyle}

\declaretheorem[style=defstyle, title=Exercise]{exo}
%________________________________________________________



%----- ENVIRONNEMENT POUR LES PREUVES ----%
\declaretheoremstyle[
  spaceabove=0pt, spacebelow=0pt, headfont=\normalfont\bfseries\scshape,
    notefont=\mdseries, notebraces={(}{)}, headpunct={. }, headindent={},
    postheadspace={ }, postheadspace=4pt, bodyfont=\normalfont, 
    mdframed={
      leftmargin=15,
      rightmargin=15,
      hidealllines=true,
      font=\small
   }
]{preuvestyle}

\declaretheorem[style=preuvestyle, numbered = no, title=Solution, qed=\qedsymbol]{sol}

%________________________________________________________


\addtokomafont{disposition}{\normalfont\bfseries}

\title{\normalfont{\bfseries{Machine Learning: Homework 2}}}
\author{Laurent \textsc{Hayez}}
\date{\today}

\clearpairofpagestyles                 % deletes header/footer
\pagestyle{scrheadings}           % use following definitions for header/footer
% definitions/configuration for the header
\rohead[Université de \textsc{Neuchâtel}]{Université de \textsc{Neuchâtel}}
\rehead[Université de \textsc{Neuchâtel}]{Université de \textsc{Neuchâtel}}        % equal page, right position (inner) 
\lohead[Laurent \textsc{Hayez}]{Laurent \textsc{Hayez}}        % odd   page, left  position (inner) 
\lehead[Laurent \textsc{Hayez}]{Laurent \textsc{Hayez}} % equal page, left (outer) position
% definitions/configuration for the footer
\lefoot[Machine Learning: Homework 2]{Machine Learning: Homework 2}
\lofoot[Machine Learning: Homework 2]{Machine Learning: Homework 2}
\refoot[page \pagemark]{page \pagemark}
\rofoot[page \pagemark]{page \pagemark}

% for compatibility stuff
\pgfplotsset{compat=1.12}


\begin{document}


\renewcommand{\labelitemi}{\textbullet}



\maketitle




\begin{exo}
  Find out (by hand or WEKA) with the simple rule (1R) which attribute best predicts whether a car gets stolen or not.
\end{exo}

  \begin{sol}
    Using the file \texttt{ML\_hayezl\_carRelation.arff} and WEKA, we derive the following one rule:
    \[ \text{\texttt{price=low $\implies$ FALSE, price=medium $\implies$ FALSE, price=high $\implies$
          TRUE.}} \]
    The details can be found in the file \texttt{ML\_hayezl\_carRelationResults.txt}.
  \end{sol}




\begin{exo}
  Decide (by hand and WEKA) whether Logan is Scottish based on the following attributes and using a Naïve
  Bayes classifier. Logan likes shortbread, drinks whiskey and eats porridge but doesn’t like lager and
  doesn’t watch soccer. Bonus: use a smoothing technique. (+1p)
\end{exo}

  \begin{sol}
    Let $E = \{$shortbread = yes, lager = no, whisky = yes, porridge = yes, soccer = no$\}$. We need to
    compute $\P[$Logan is Scottish $\mid E]$. With the naive Bayes approach, we need to compute
    \begin{align*}
      \P[\text{Logan is Scottish} \mid E] = \frac{1}{\P[E]} \cdot &\P[\text{shortbread = yes} \mid yes] \cdot \\
      & \P[\text{lager = no} \mid yes] \cdot \\
      & \P[\text{whisky = yes} \mid yes] \cdot \\
      & \P[\text{porridge = yes} \mid yes] \cdot \\
      & \P[\text{soccer = no} \mid yes] \cdot \\
      & \P[\text{being Scottish}] \\
    \end{align*}

    Computing the conditional probabilities and the probability of being Scottish using Table 1, we obtain 
    \[ \frac{6}{3} \cdot \frac{3}{7} \cdot \frac{4}{7} \cdot \frac{5}{7} \cdot \frac{4}{7} \cdot \frac{7}{13}
      = 0.046. \]
    Computing the same thing but for $\P[\text{Logan is not Scottish} \mid E]$, we obtain $0.0064$. Hence 
    \[ \P[E] = 0.046 + 0.0064 = 0.052, \]
    \[ \P[\text{Logan is Scottish} \mid E] = 0.878, \]
    \[ \P[\text{Logan is not Scottish} \mid E] = 0.122. \]
    Hence Logan is Scottish with probability $0.878$.

    We can use for example the Laplace estimator as a smoothing technique so that we don't have conditional
    probabilities being equal to 0. The Laplace estimator add 1 to the count for every attribute value class
    combination. Hence Table 1 becomes Table 2. The previous computations become $\P[E] = 0.038 + 0.008 =
    0.046$, 
    \[ \P[\text{Logan is Scottish} \mid E] = 0.825 \]
    \[ \P[\text{Logan is not Scottish} \mid E] = 0.174 \]
    so the conclusion does not change. 

    WEKA gives the same result by training it with the file \texttt{ML\_hayezl\_scottsRelation.arff} and
    predicting if Logan is Scottish or not with the file \texttt{ML\_hayezl\_scottsRelationPredict.arff}. The
    results can be found in \texttt{ML\_hayezl\_scottsRelationResults.txt} for the training and in
    \texttt{ML\_hayezl\_scottsRelationPredictResults.txt} for the prediction. 
  \end{sol}


\begin{table}[]
\centering
\caption{Probabilities of the attributes given that the person is Scottish or not}
\begin{tabular}{@{}lll|lll|lll@{}}
\toprule
Shortbread & yes & no  & Lager  & yes & no  & Whisky & yes & no  \\ \midrule
yes        & 6   & 3   & yes    & 4   & 3   & yes    & 4   & 2   \\
no         & 1   & 3   & no     & 3   & 3   & no     & 3   & 4   \\
yes        & 6/7 & 3/6 & yes    & 4/7 & 3/6 & yes    & 4/7 & 2/6 \\
no         & 1/7 & 3/6 & no     & 3/7 & 3/6 & no     & 3/7 & 4/6 \\ \midrule
Porridge   & yes & no  & Soccer & yes & no  &        &     &     \\ \cmidrule(r){1-6}
yes        & 5   & 3   & yes    & 3   & 4   &        &     &     \\
no         & 2   & 3   & no     & 4   & 2   &        &     &     \\
yes        & 5/7 & 3/6 & yes    & 3/7 & 4/6 &        &     &     \\
no         & 2/7 & 3/6 & no     & 4/7 & 2/6 &        &     &     \\ \bottomrule
\end{tabular}
\end{table}


\begin{table}[]
\centering
\caption{Probabilities of the attributes given that the person is Scottish or not (with Laplace estimator)}
\begin{tabular}{@{}lll|lll|lll@{}}
\toprule
Shortbread & yes & no  & Lager  & yes & no  & Whisky & yes & no  \\ \midrule
yes        & 7   & 4   & yes    & 5   & 4   & yes    & 5   & 3   \\
no         & 2   & 4   & no     & 4   & 4   & no     & 4   & 5   \\
yes        & 7/9 & 4/8 & yes    & 5/9 & 4/8 & yes    & 5/9 & 3/8 \\
no         & 2/9 & 4/8 & no     & 4/9 & 4/8 & no     & 4/9 & 5/8 \\ \midrule
Porridge   & yes & no  & Soccer & yes & no  &        &     &     \\ \cmidrule(r){1-6}
yes        & 6   & 4   & yes    & 4   & 5   &        &     &     \\
no         & 3   & 4   & no     & 5   & 3   &        &     &     \\
yes        & 6/9 & 4/8 & yes    & 4/9 & 5/8 &        &     &     \\
no         & 3/9 & 4/8 & no     & 5/9 & 3/8 &        &     &     \\ \bottomrule
\end{tabular}
\end{table}


	
\end{document}



%%% Local Variables:
%%% mode: latex
%%% TeX-master: t 
%%% End: