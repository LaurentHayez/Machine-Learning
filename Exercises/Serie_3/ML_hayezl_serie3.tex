\documentclass[fontsize=12pt, usenames, dvipsnames, headinclude, headsepline, footinclude, footsepline]{scrartcl}

\usepackage[utf8]{inputenc}
\usepackage[T1]{fontenc}
\usepackage{graphicx, wrapfig}
\usepackage{lmodern}
\usepackage{color, colortbl}
\usepackage{xcolor}
\usepackage{amsmath, amssymb, mathrsfs, amsthm, thmtools, MnSymbol}
\usepackage[framemethod=tikz]{mdframed}
\usepackage{pgf, pgfplots, tikz, pst-solides3d}
\usetikzlibrary{cd} %To draw commutative diagrams
\usetikzlibrary{calc}
\usetikzlibrary{arrows}
\usetikzlibrary{shapes}
\usetikzlibrary{lindenmayersystems, arrows.meta}
\usetikzlibrary{decorations.pathreplacing}
\usetikzlibrary{automata}
\usepackage{scrlayer-scrpage}  % header and footer for KOMA-Script
\usepackage{hyperref}
\usepackage{todonotes}
\usepackage{listings}
\usepackage[inline]{enumitem}
\usepackage{booktabs}
\usepackage{multirow}
\usepackage[english]{babel}


\newcommand{\N}{\mathbb{N}}
\newcommand{\Q}{\mathbb{Q}} 
\newcommand{\R}{\mathbb{R}}
\newcommand{\Z}{\mathbb{Z}}
\newcommand{\F}{\mathbb{F}}
\newcommand{\T}{\mathcal{T}}
\renewcommand{\P}{\mathbb{P}}
\renewcommand{\S}{\mathbb{S}}
\newcommand{\bw}{\bigwedge}
\newcommand{\Fa}{\F(A)} 
\newcommand{\C}{\mathbb{C}}
\newcommand{\K}{\mathbb{K}}
\renewcommand{\epsilon}{\varepsilon}
\renewcommand{\phi}{\varphi}
\renewcommand{\emph}{\textbf}
\newcommand{\im}{\mathrm{im}}


\synctex=1


%%%%%%%%	Définitions des environnements de théorèmes	%%%%%%%%
%----- ENVIRONNEMENT POUR LES EXERCICES ----%
\declaretheoremstyle[
  spaceabove=0pt, spacebelow=0pt, headfont=\normalfont\bfseries\scshape,
    notefont=\mdseries, notebraces={(}{)}, headpunct={. }, headindent={},
    postheadspace={ }, postheadspace=4pt, bodyfont=\normalfont\itshape
]{defstyle}

\declaretheorem[style=defstyle, title=Exercise]{exo}
%________________________________________________________



%----- ENVIRONNEMENT POUR LES PREUVES ----%
\declaretheoremstyle[
  spaceabove=0pt, spacebelow=0pt, headfont=\normalfont\bfseries\scshape,
    notefont=\mdseries, notebraces={(}{)}, headpunct={. }, headindent={},
    postheadspace={ }, postheadspace=4pt, bodyfont=\normalfont, 
    mdframed={
      leftmargin=15,
      rightmargin=15,
      hidealllines=true,
      font=\small
   }
]{preuvestyle}

\declaretheorem[style=preuvestyle, numbered = no, title=Solution, qed=\qedsymbol]{sol}

%________________________________________________________


\addtokomafont{disposition}{\normalfont\bfseries}

\title{\normalfont{\bfseries{Machine Learning: Homework 3}}}
\author{Laurent \textsc{Hayez}}
\date{\today}

\clearpairofpagestyles                 % deletes header/footer
\pagestyle{scrheadings}           % use following definitions for header/footer
% definitions/configuration for the header
\rohead[Université de \textsc{Neuchâtel}]{Université de \textsc{Neuchâtel}}
\rehead[Université de \textsc{Neuchâtel}]{Université de \textsc{Neuchâtel}}        % equal page, right position (inner) 
\lohead[Laurent \textsc{Hayez}]{Laurent \textsc{Hayez}}        % odd   page, left  position (inner) 
\lehead[Laurent \textsc{Hayez}]{Laurent \textsc{Hayez}} % equal page, left (outer) position
% definitions/configuration for the footer
\lefoot[Machine Learning: Homework 3]{Machine Learning: Homework 3}
\lofoot[Machine Learning: Homework 3]{Machine Learning: Homework 3}
\refoot[page \pagemark]{page \pagemark}
\rofoot[page \pagemark]{page \pagemark}

% for compatibility stuff
\pgfplotsset{compat=1.12}


\begin{document}


\renewcommand{\labelitemi}{\textbullet}



\maketitle




\begin{exo}
  Invent the two characters Taylor and Robin each with a weight in the range of [63-74] kg and a shoe size in
  the range of [40-44]. Decide (by hand) with Naïve Bayes using the probability density function whether your
  Taylor and Robin are female or male based on the following statistics.
\end{exo}

  \begin{sol}
    As we have to deal with numerical attributes, we need to use a probability density function 
    \[ f(x) = \frac{1}{\sigma \sqrt{2 \pi}} e^{-\frac{(x - \mu)^2}{2\sigma^2}} \]
    where $\mu$ is the sample mean and $\sigma$ is the standard deviation.

    In Table 1, we represented the weight and shoe size according to whether the person is male or female, and
    we computed the sample means and standard deviations. 

    Let $f_1$ be a gaussian distribution with $\mu = \mu_1$ and $\sigma = \sigma_1$ representing the
    distribution of the weight for the males. Let $g_1$ be the same but with $\mu = \mu_2$ and $\sigma =
    \sigma_2$ representing the distribution of the weight for the females. In the same fashion we define $f_2$
    and $g_2$ corresponding to the shoe size. Moreover let $L[\cdot]$ denote the likelihood of an event.

    If Taylor weights $73$ kgs and has shoe size 41, and Robin weights $67$ kgs and has shoe size $44$, then
    define $E_1 = \{\text{weight = 73, shoe size = 41}\}$ and $E_2 = \{\text{weight = 67, shoe size =
      44}\}$. We have
    \begin{align*}
      L[\text{Taylor = male} \mid E_1] &= f_1(73) \cdot f_2(41) \cdot \frac{1}{2} & \text{where }
                                                                                    \P[\text{male}] =
                                                                                    \frac{1}{2}\\
                                       &= 0.0505 \cdot 0.0118 \cdot \frac{1}{2}\\
      &= 0.00029
    \end{align*}
  \end{sol}







	
\end{document}



%%% Local Variables:
%%% mode: latex
%%% TeX-master: t 
%%% End: