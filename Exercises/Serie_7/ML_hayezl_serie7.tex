\documentclass[fontsize=12pt, usenames, dvipsnames, headinclude, headsepline, footinclude, footsepline]{scrartcl}

\usepackage[utf8]{inputenc}
\usepackage[T1]{fontenc}
\usepackage{graphicx, wrapfig}
\usepackage{lmodern}
\usepackage{color, colortbl}
\usepackage{xcolor}
\usepackage{amsmath, amssymb, mathrsfs, amsthm, thmtools, MnSymbol}
\usepackage[framemethod=tikz]{mdframed}
\usepackage{pgf, pgfplots, tikz, pst-solides3d}
\usepackage{scrlayer-scrpage}  % header and footer for KOMA-Script
\usepackage{hyperref}
\usepackage{todonotes}
\usepackage{listings}
\usepackage[inline]{enumitem}
\usepackage{booktabs}
\usepackage{verbatim, listings}
\usepackage{multirow}
\usepackage{pdflscape}
\usepackage[english]{babel}


\newcommand{\N}{\mathbb{N}}
\newcommand{\Q}{\mathbb{Q}} 
\newcommand{\R}{\mathbb{R}}
\newcommand{\Z}{\mathbb{Z}}
\newcommand{\F}{\mathbb{F}}
\newcommand{\T}{\mathcal{T}}
\renewcommand{\P}{\mathbb{P}}
\renewcommand{\S}{\mathbb{S}}
\newcommand{\bw}{\bigwedge}
\newcommand{\Fa}{\F(A)} 
\newcommand{\C}{\mathbb{C}}
\newcommand{\K}{\mathbb{K}}
\renewcommand{\epsilon}{\varepsilon}
\renewcommand{\phi}{\varphi}
\renewcommand{\emph}{\textbf}
\newcommand{\im}{\mathrm{im}}


\synctex=1


%%%%%%%%	Définitions des environnements de théorèmes	%%%%%%%%
%----- ENVIRONNEMENT POUR LES EXERCICES ----%
\declaretheoremstyle[
  spaceabove=0pt, spacebelow=0pt, headfont=\normalfont\bfseries\scshape,
    notefont=\mdseries, notebraces={(}{)}, headpunct={. }, headindent={},
    postheadspace={ }, postheadspace=4pt, bodyfont=\normalfont\itshape
]{defstyle}

\declaretheorem[style=defstyle, title=Exercise]{exo}
%________________________________________________________



%----- ENVIRONNEMENT POUR LES PREUVES ----%
\declaretheoremstyle[
  spaceabove=0pt, spacebelow=0pt, headfont=\normalfont\bfseries\scshape,
    notefont=\mdseries, notebraces={(}{)}, headpunct={. }, headindent={},
    postheadspace={ }, postheadspace=4pt, bodyfont=\normalfont, 
    mdframed={
      leftmargin=15,
      rightmargin=15,
      hidealllines=true,
      font=\small
   }
]{preuvestyle}

\declaretheorem[style=preuvestyle, numbered = no, title=Solution, qed=\qedsymbol]{sol}
%________________________________________________________


\addtokomafont{disposition}{\normalfont\bfseries}

\title{\normalfont{\bfseries{Machine Learning: Homework 7}}}
\author{Laurent \textsc{Hayez}}
\date{\today}

\clearpairofpagestyles                 % deletes header/footer
\pagestyle{scrheadings}           % use following definitions for header/footer
% definitions/configuration for the header
\rohead[Université de \textsc{Neuchâtel}]{Université de \textsc{Neuchâtel}}
\rehead[Université de \textsc{Neuchâtel}]{Université de \textsc{Neuchâtel}}        % equal page, right position (inner) 
\lohead[Laurent \textsc{Hayez}]{Laurent \textsc{Hayez}}        % odd   page, left  position (inner) 
\lehead[Laurent \textsc{Hayez}]{Laurent \textsc{Hayez}} % equal page, left (outer) position
% definitions/configuration for the footer
\lefoot[Machine Learning: Homework 7]{Machine Learning: Homework 7}
\lofoot[Machine Learning: Homework 7]{Machine Learning: Homework 7}
\refoot[page \pagemark]{page \pagemark}
\rofoot[page \pagemark]{page \pagemark}


\begin{document}


\renewcommand{\labelitemi}{\textbullet}



\maketitle




\begin{exo}
  Build the co-occurrence matrix for the observations.

Calculate support, confidence, completeness, lift, and leverage for the following rules.
\begin{itemize}
\item Apple $\to$ Donut 
\item Apple $\to$ Onion 
\item Sugar $\to$ Yoghurt 
\item Donut $\to$ Onion 
\item Donut $\to$ Raspberry 
\item Onion $\to$ Raspberry
\end{itemize}

Explain these measures (why they are useful, what ranges of numbers they can return, what the values mean).

Use the Apriori algorithm to find frequent item sets. We are only interested in item sets having a support value of at least 50\%.

\end{exo}

  \begin{sol}
    The co-occurrence matrix is computed in \texttt{ML\_hayezl\_homework7.xlsx}, and is
    given in Table \ref{table:co-occ}. 

    The support, confidence, completeness, lift and leverage were computed with the following formulas 
    \[ \mathrm{support} = \frac{N_{\mathrm{both}}}{N_{\mathrm{total}}},\] 
    \[ \mathrm{confidence} = \frac{N_{\mathrm{both}}}{N_{\mathrm{left}}}, \] 
    \[ \mathrm{completeness} = \frac{N_{\mathrm{both}}}{N_{\mathrm{right}}}, \] 
    \[ \mathrm{lift}(L \to R) = \frac{\mathrm{support}(L \cup R)}{\mathrm{support}(L) \cdot
        \mathrm{support}(R)},  \] 
    \[ \mathrm{leverage}(L \to R) =  \mathrm{support}(L \cup R) - (\mathrm{support}(L) \cdot
      \mathrm{support}(R)).\]
    The results for the different rules are displayed in Table \ref{table:1}.

    \begin{itemize}
    \item The \emph{support} of a set $I$ measures the proportion of baskets in which $I$ appears. Hence
      $\mathrm{support}(I) \in [0, 1]$ (or $]0, 1]$ to be more precise, because considering items that never
      appear is not interesting). This measure is useful to know if $I$ appears often or not.

    \item The \emph{confidence} of a rule measures how reliable a rule is, or in other words, if the rule is
      $L \to R$, it measures the proportion of the appearance of $R$ when $L$ appears. This measure takes
      values in $]0,1]$. It is useful to know if some items are correlated, or often bought together. 

    \item The \emph{completeness} of a rule measures the proportion of times $L$ and $R$ happen with respect
      to $R$. This measure takes values in $]0,1]$. If the measure is close to $1$, it means that $R$ is very
      correlated with $L$, as it means that $R$ appears almost always when $L$ appears. It is useful to
      determine if an item is bought when another item is bought.

    \item The \emph{lift} of a rule $L \to R$ measure how the appearance of two items at the same time differ
      from how they would appear if $L$ and $R$ were statistically independent. If $L$ and $R$ are
      independent, the expectation of $L$ and $R$ appearing together is $|L| \cdot \mathrm{support}(R)$, and
      we need to compare this to the actual number of time they appear together, i.e., $|L \cup R|$. This
      measure takes values in $\R_{>0}$, but the interesting values are when $\mathrm{lift}(L \to R) > 1$
      because this tells us that $L$ and $R$ are correlated, in the sense that when $L$ is bought, $R$ is also
      bought. 

    \item The \emph{leverage} of a rule $L \to R$ compares the support of $L \cup R$ and $L$, $R$. It gives a
      measure that tells us whether the elements are associated ``by chance''. This measure takes values in
      $R_{>0}$. It measures the proportion of times items are bought together more than if we had chosen them
      randomly. 
    \end{itemize}

    We start by creating
    $L_1 = \{\{\text{Apple}\}, \{\text{Donut}\}, \{\text{Ice-cream}\}, \{\text{Onion}\},
    \{\text{Raspberry}\}\}$ which consists of the items that have a support at least 50\%. From this set we
    create $C_2$ which consist of the 10 possible unordered pairs of items. We keep the pairs that have a
    support greater than 50\% and we create 
    \begin{align*}L_2 = \{&\{\text{Apple, Donut}\}, \{\text{Apple, Ice-cream}\}, \{\text{Apple, Onion}\},\\ & \{\text{Apple,
      Raspberry}\}, \{\text{Donut, Onion}\}, \{\text{Onion, Raspberry}\}\}.\end{align*}
    From $L_2$ we can form $C_3 = \{\{\text{A, D, I}\}, \{\text{A, D, O}\}, \{\text{A, D, R}\}, \{\text{D, O,
      R}\}\}$ where $A = $ Apple, $D = $ Donut, $I = $ Ice-cream, $O = $ Onion, $R = $ Raspberry. The only set
    with support greater than 50\% is $\{\text{Apple, Donut, Onions}\} =: L_3$, and we can't form any other set.
  \end{sol}


% Please add the following required packages to your document preamble:
% \usepackage{booktabs}
\begin{table}[h]
\centering
\caption{Co-occurrence matrix for the observation}
\label{table:co-occ}
{ \footnotesize 
\begin{tabular}{@{}c|ccccccccc@{}}
\toprule
Co-occurrences & Apple & Donut & Ice-cream & Mango & Onion & Raspberry & Sugar & Tomato & Yoghurt \\ \midrule
Apple          & 6     & 4     & 3         & 2     & 5     & 4         & 2     & 1      & 2       \\
Donut          & 4     & 4     & 2         & 0     & 3     & 2         & 2     & 1      & 1       \\
Ice-cream      & 3     & 2     & 3         & 1     & 2     & 2         & 1     & 0      & 1       \\
Mango          & 2     & 0     & 1         & 2     & 2     & 2         & 0     & 0      & 1       \\
Onion          & 5     & 3     & 2         & 2     & 5     & 4         & 1     & 1      & 1       \\
Raspberry      & 4     & 2     & 2         & 2     & 4     & 4         & 1     & 0      & 1       \\
Sugar          & 2     & 2     & 1         & 0     & 1     & 1         & 2     & 0      & 1       \\
Tomato         & 1     & 1     & 0         & 0     & 1     & 0         & 0     & 1      & 0       \\
Yoghurt        & 2     & 1     & 1         & 1     & 1     & 1         & 1     & 0      & 2       \\ \bottomrule
\end{tabular}
}
\end{table}

% Please add the following required packages to your document preamble:
% \usepackage{booktabs}
\begin{table}[h]
\centering
\caption{Support, confidence, completeness, lift and leverage of the different rules}
\label{table:1}
{ \footnotesize
\begin{tabular}{@{}c|ccc@{}}
\toprule
Rules        & Apple $\to$ Donut & Apple $\to$ Onion     & Sugar $\to$ Yoghurt   \\ \midrule
Support      & 0.666666667               & 0.833333333                   & 0.166666667                   \\
Confidence   & 0.666666667               & 0.833333333                   & 0.5                           \\
Completeness & 1                         & 1                             & 0.5                           \\
Lift         & 1                         & 1                             & 1.5                           \\
Leverage     & 0                         & 0                             & 0.055555556                   \\ \toprule
Rules        & Donut $\to$ Onion & Donut $\to$ Raspberry & Onion $\to$ Raspberry \\ \midrule
Support      & 0.5                       & 0.333333333                   & 0.666666667                   \\
Confidence   & 0.75                      & 0.5                           & 0.8                           \\
Completeness & 0.6                       & 0.5                           & 1                             \\
Lift         & 0.9                       & 0.75                          & 1.2                           \\
Leverage     & -0.055555556              & -0.111111111                  & 0.111111111                   \\ \bottomrule
\end{tabular}
}
\end{table}

	
\end{document}



%%% Local Variables:
%%% mode: latex
%%% TeX-master: t 
%%% End: