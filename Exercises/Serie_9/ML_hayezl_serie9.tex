\documentclass[fontsize=12pt, usenames, dvipsnames, headinclude, headsepline, footinclude, footsepline]{scrartcl}

\usepackage[utf8]{inputenc}
\usepackage[T1]{fontenc}
\usepackage{graphicx, wrapfig}
\usepackage{lmodern}
\usepackage{color, colortbl}
\usepackage{xcolor}
\usepackage{amsmath, amssymb, mathrsfs, amsthm, thmtools, MnSymbol}
\usepackage[framemethod=tikz]{mdframed}
\usepackage{pgf, pgfplots, tikz, pst-solides3d}
\usepackage{scrlayer-scrpage}  % header and footer for KOMA-Script
\usepackage{hyperref}
\usepackage{todonotes}
\usepackage{listings}
\usepackage[inline]{enumitem}
\usepackage{booktabs}
\usepackage{verbatim, listings}
\usepackage{multirow}
\usepackage{pdflscape}
\usepackage[english]{babel}


\newcommand{\N}{\mathbb{N}}
\newcommand{\Q}{\mathbb{Q}} 
\newcommand{\R}{\mathbb{R}}
\newcommand{\Z}{\mathbb{Z}}
\newcommand{\F}{\mathbb{F}}
\newcommand{\T}{\mathcal{T}}
\renewcommand{\P}{\mathbb{P}}
\renewcommand{\S}{\mathbb{S}}
\newcommand{\bw}{\bigwedge}
\newcommand{\Fa}{\F(A)} 
\newcommand{\C}{\mathbb{C}}
\newcommand{\K}{\mathbb{K}}
\renewcommand{\epsilon}{\varepsilon}
\renewcommand{\phi}{\varphi}
\renewcommand{\emph}{\textbf}
\newcommand{\im}{\mathrm{im}}


\synctex=1


%%%%%%%%	Définitions des environnements de théorèmes	%%%%%%%%
%----- ENVIRONNEMENT POUR LES EXERCICES ----%
\declaretheoremstyle[
  spaceabove=0pt, spacebelow=0pt, headfont=\normalfont\bfseries\scshape,
    notefont=\mdseries, notebraces={(}{)}, headpunct={. }, headindent={},
    postheadspace={ }, postheadspace=4pt, bodyfont=\normalfont\itshape
]{defstyle}

\declaretheorem[style=defstyle, title=Exercise]{exo}
%________________________________________________________



%----- ENVIRONNEMENT POUR LES PREUVES ----%
\declaretheoremstyle[
  spaceabove=0pt, spacebelow=0pt, headfont=\normalfont\bfseries\scshape,
    notefont=\mdseries, notebraces={(}{)}, headpunct={. }, headindent={},
    postheadspace={ }, postheadspace=4pt, bodyfont=\normalfont, 
    mdframed={
      leftmargin=15,
      rightmargin=15,
      hidealllines=true,
      font=\small
   }
]{preuvestyle}

\declaretheorem[style=preuvestyle, numbered = no, title=Solution, qed=\qedsymbol]{sol}
%________________________________________________________


\addtokomafont{disposition}{\normalfont\bfseries}

\title{\normalfont{\bfseries{Machine Learning: Homework 8}}}
\author{Laurent \textsc{Hayez}}
\date{\today}

\clearpairofpagestyles                 % deletes header/footer
\pagestyle{scrheadings}           % use following definitions for header/footer
% definitions/configuration for the header
\rohead[Université de \textsc{Neuchâtel}]{Université de \textsc{Neuchâtel}}
\rehead[Université de \textsc{Neuchâtel}]{Université de \textsc{Neuchâtel}}        % equal page, right position (inner) 
\lohead[Laurent \textsc{Hayez}]{Laurent \textsc{Hayez}}        % odd   page, left  position (inner) 
\lehead[Laurent \textsc{Hayez}]{Laurent \textsc{Hayez}} % equal page, left (outer) position
% definitions/configuration for the footer
\lefoot[Machine Learning: Homework 8]{Machine Learning: Homework 8}
\lofoot[Machine Learning: Homework 8]{Machine Learning: Homework 8}
\refoot[page \pagemark]{page \pagemark}
\rofoot[page \pagemark]{page \pagemark}


\begin{document}


\renewcommand{\labelitemi}{\textbullet}



\maketitle




\begin{exo}
  We have the following (fictional) language similarity matrix:

  \begin{center}
    \begin{tabular}{@{}c|cccccc@{}}
      \toprule
      & Czech & Polish & Russian & English & Danish & Swedish \\ \midrule
      Czech   &       & 0.85   & 0.7     & 0.3     & 0.25   & 0.2     \\
      Polish  & 0.85  &        & 0.4     & 0.25    & 0.7    & 0.8     \\
      Russian & 0.7   & 0.4    &         & 0.3     & 0.1    & 0.2     \\
      English & 0.3   & 0.25   & 0.3     &         & 0.75   & 0.8     \\
      Danish  & 0.25  & 0.7    & 0.1     & 0.75    &        & 0.95    \\
      Swedish & 0.2   & 0.8    & 0.2     & 0.8     & 0.95   &         \\ \bottomrule
    \end{tabular}
  \end{center}

  Do clustering (by hand) using two different techniques, once a complete link agglomerative clustering, and
  once a single link agglomerative clustering. Illustrate the intermediate steps and draw the final
  dendrograms.
\end{exo}

  \begin{sol}
    \begin{enumerate}
    \item We start with the complete link method.

      \begin{center}
        \begin{tabular}{@{}c|cccccc@{}}
          \toprule
          & Czech & Polish & Russian & English & Danish & Swedish \\ \midrule
          Czech   &       & 0.85   & 0.7     & 0.3     & 0.25   & 0.2     \\
          Polish  & 0.85  &        & 0.4     & 0.25    & 0.7    & 0.8     \\
          Russian & 0.7   & 0.4    &         & 0.3     & \cellcolor{ForestGreen!30}0.1    & 0.2     \\
          English & 0.3   & 0.25   & 0.3     &         & 0.75   & 0.8     \\
          Danish  & 0.25  & 0.7    & 0.1     & 0.75    &        & 0.95    \\
          Swedish & 0.2   & 0.8    & 0.2     & 0.8     & 0.95   &         \\ \bottomrule
        \end{tabular}
      \end{center}

      We construct the next table according to the following rule 
      \[ \mathrm{sim}(c_i, c_j) = \min_{x \in c_i,\ y \in c_j}\{\mathrm{sim}(x, y)\}. \]
      
      \begin{center}
        \begin{tabular}{@{}c|ccccc@{}}
          \toprule
          & Czech & Polish & Russian, Danish & English & Swedish \\ \midrule
          Czech           &       & 0.85   & 0.25            & 0.3     & 0.2     \\
          Polish          & 0.85  &        & 0.4             & 0.25    & 0.8     \\
          Russian, Danish & 0.25  & 0.4    &                 & 0.3     & \cellcolor{ForestGreen!30}0.2     \\
          English         & 0.3   & 0.25   & 0.3             &         & 0.8     \\
          Swedish         & 0.2   & 0.8    & 0.2             & 0.8     &         \\ \bottomrule
        \end{tabular}
      \end{center}
      
      \begin{center}
        \begin{tabular}{@{}c|cccc@{}}
          \toprule
          & Czech & Polish & Russian, Danish, Swedish & English \\ \midrule
          Czech                    &       & 0.85   & \cellcolor{ForestGreen!30}0.2                      & 0.3     \\
          Polish                   & 0.85  &        & 0.4                      & 0.25    \\
          Russian, Danish, Swedish & 0.2   & 0.4    &                          & 0.3     \\
          English                  & 0.3   & 0.25   & 0.3                      &         \\ \bottomrule
        \end{tabular}
      \end{center}

      \begin{center}
        \begin{tabular}{@{}c|ccc@{}}
          \toprule
          & Polish & Rus, Dan, Swe, Cze & English \\ \midrule
          Polish             &        & 0.4                & \cellcolor{ForestGreen!30}0.25    \\
          Rus, Dan, Swe, Cze & 0.4    &                    & 0.3     \\
          English            & 0.25   & 0.3                &         \\ \bottomrule
        \end{tabular}
      \end{center}

      \begin{center}
        \begin{tabular}{@{}c|cc@{}}
          \toprule
          & Pol, En & Rus, Dan, Swe, Cze \\ \midrule
          Pol, En            &         & 0.3                \\
          Rus, Dan, Swe, Cze & 0.3     &                    \\ \bottomrule
        \end{tabular}
      \end{center}
      
      The resulting dendrogram is the following.
      
      \begin{center}
        \begin{tikzpicture}
          \draw (-2, 0) -- (-2, 4);
          \foreach \i in {0, 1, 2, 3} {
            \draw (-2.1, \i) node[left]{$0.\i$} -- (-1.9, \i);
          }
          \node (1) at (-1, 0) {Rus};
          \node (2) at (0,0) {Dan};
          \node (3) at (1, 0) {Swe};
          \node (4) at (2, 0) {Cze};
          \node (5) at (3, 0) {Pol};
          \node (6) at (4, 0) {En};
          \draw (1) -- (-1, 1) -- (0, 1) -- (2);
          \draw (-0.5, 1) -- (-0.5, 2) -- (2, 2) -- (4) (3) -- (1, 3);
          \draw (1, 3) -- (3, 3) -- (5) (3, 3) -- (4, 3) -- (6);
          \draw (2.5, 3) -- (2.5, 4);
        \end{tikzpicture}
      \end{center}


    \item We do the same but with the single link. 

      \begin{center}
        \begin{tabular}{@{}c|cccccc@{}}
          \toprule
          & Czech & Polish & Russian & English & Danish & Swedish \\ \midrule
          Czech   &       & 0.85   & 0.7     & 0.3     & 0.25   & 0.2     \\
          Polish  & 0.85  &        & 0.4     & 0.25    & 0.7    & 0.8     \\
          Russian & 0.7   & 0.4    &         & 0.3     & 0.1    & 0.2     \\
          English & 0.3   & 0.25   & 0.3     &         & 0.75   & 0.8     \\
          Danish  & 0.25  & 0.7    & 0.1     & 0.75    &        & \cellcolor{ForestGreen!30}0.95    \\
          Swedish & 0.2   & 0.8    & 0.2     & 0.8     & 0.95   &         \\ \bottomrule
        \end{tabular}
      \end{center}

      We construct the next table according to the following rule 
      \[ \mathrm{sim}(c_i, c_j) = \max_{x \in c_i,\ y \in c_j}\{\mathrm{sim}(x, y)\}. \]

      \begin{center}
        \begin{tabular}{@{}l|ccccc@{}}
          \toprule
          & Czech & Polish & Russian & English & Dan, Swe \\ \midrule
          Czech    &       & \cellcolor{ForestGreen!30} 0.85   & 0.7     & 0.3     & 0.25     \\
          Polish   & 0.85  &        & 0.4     & 0.25    & 0.8      \\
          Russian  & 0.7   & 0.4    &         & 0.3     & 0.2      \\
          English  & 0.3   & 0.25   & 0.3     &         & 0.8      \\
          Dan, Swe & 0.8   & 0.8    & 0.2     & 0.8     &          \\ \bottomrule
        \end{tabular}
      \end{center}

      \begin{center}
        \begin{tabular}{@{}l|cccc@{}}
          \toprule
          & Cze, Pol & Russian & English & Dan, Swe \\ \midrule
          Cze, Pol &          & 0.7     & 0.3     & \cellcolor{ForestGreen!30}0.8      \\
          Russian  & 0.7      &         & 0.3     & 0.2      \\
          English  & 0.3      & 0.3     &         & 0.8      \\
          Dan, Swe & 0.8      & 0.2     & 0.8     &          \\ \bottomrule
        \end{tabular}
      \end{center}

      \begin{center}
        \begin{tabular}{@{}l|ccc@{}}
          \toprule
          & Cze, Pol, Dan, Swe & Russian & English \\ \midrule
          Cze, Pol, Dan, Swe &                    & 0.7     & \cellcolor{ForestGreen!30}0.8     \\
          Russian            & 0.7                &         & 0.3     \\
          English            & 0.8                & 0.3     &         \\ \bottomrule
        \end{tabular}
      \end{center}

      \begin{center}
        \begin{tabular}{@{}l|cc@{}}
          \toprule
          & Cze, Pol, Dan, Swe, En & Russian \\ \midrule
          Cze, Pol, Dan, Swe, En &                        & 0.7     \\
          Russian                & 0.7                    &         \\ \bottomrule
        \end{tabular}
      \end{center}

      The resulting dendrogram is the following.

      \begin{center}
        \begin{tikzpicture}
          \draw (-2, 0) -- (-2, 4);
          \draw (-2.1, 0) node[left] {1} -- (-1.9, 0);
          \draw (-2.1, 1) node[left] {0.9} -- (-1.9, 1);
          \draw (-2.1, 2) node[left] {0.8} -- (-1.9, 2);
          \draw (-2.1, 3) node[left] {0.7} -- (-1.9, 3);
          \node (DAN) at (-1, 0) {Dan};
          \node (SWE) at (0,0) {Swe};
          \node (CZE) at (1, 0) {Cze};
          \node (POL) at (2, 0) {Pol};
          \node (EN) at (3, 0) {En};
          \node (RUS) at (4, 0) {Rus};
          \draw (DAN) -- (-1, 0.5) -- (0, 0.5) -- (SWE);
          \draw (-0.5, 0.5) -- (-0.5, 2) -- (3, 2) -- (EN);
          \draw (1.5, 2) -- (1.5, 1.5) -- (2, 1.5) -- (POL) (1.5, 1.5) -- (1, 1.5) -- (CZE);
          \draw (0.5, 2) -- (0.5, 3) -- (4, 3) -- (RUS) (2.25, 3) -- (2.25, 4);
          \draw[dashed, gray] (-2, 0.5) -- (-1, 0.5);
          \draw[dashed, gray] (-2, 1.5) -- (1, 1.5);
        \end{tikzpicture}
      \end{center}
    \end{enumerate}
  \end{sol}




\begin{exo}
  The dataset Canton.txt shows the votes per canton (as percentage of yes) for all federal votes during the
  last three years (including topic, date, and vote number). Transform it to arff and then visualize the
  cluster tree using WEKA. Cluster all the cantons together with the average link hierarchical clustering
  applying a Manhattan distance without normalization. We want the distance between clusters to be interpreted
  as branch length and make sure leafs have the names of the cantons.  

  Submit the arff file, visualization, and WEKA output.
\end{exo}

\begin{sol}
  Command used in WEKA:
{\scriptsize
\begin{verbatim}
weka.clusterers.HierarchicalClusterer -N 1 -L AVERAGE -P -B -A "weka.core.ManhattanDistance -R first-last"
\end{verbatim}
}
\end{sol}



\end{document}



%%% Local Variables:
%%% mode: latex
%%% TeX-master: t 
%%% End: